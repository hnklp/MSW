\documentclass[a4paper,12pt]{article}
\usepackage[utf8]{inputenc}
\usepackage[T1]{fontenc}
\usepackage[czech]{babel}
\usepackage{graphicx}
\usepackage{ragged2e}
\usepackage{geometry}
\usepackage{listings}
\usepackage{titlesec}
\usepackage{fancyhdr}

\titleformat{\section}[block]{\normalfont\Large\bfseries}{}{0pt}{}
\titleformat{\subsection}[block]{\normalfont\large\bfseries}{}{0pt}{}
\titleformat{\subsubsection}[block]{\normalfont\normalsize\bfseries}{}{0pt}{}

\begin{document}
	\section {Úloha 1 - Knihovny a moduly pro matematické\\ výpočty}
	\subsection{Zadání}
V tomto kurzu jste se učili s některými vybranými knihovnami. Některé sloužily pro rychlé vektorové operace, jako numpy, některé mají naprogramovány symbolické manipulace, které lze převést na numerické reprezentace (sympy), některé mají v sobě funkce pro numerickou integraci (scipy). Některé slouží i pro rychlé základní operace s čísly (numba).
\justify
Vaším úkolem je změřit potřebný čas pro vyřešení nějakého problému (např.: provést skalární součin, vypočítat určitý integrál) pomocí standardního pythonu a pomocí specializované knihovny. Toto měření proveďte alespoň pro 5 různých úloh (ne pouze jiná čísla, ale úplně jiné téma) a minimálně porovnejte rychlost jednoho modulu se standardním pythonem. Ideálně proveďte porovnání ještě s dalším modulem a snažte se, ať je kód ve standardním pythonu napsán efektivně.  



\subsection{Řešení}
\subsubsection{1. Výpočet faktoriálu}
Tato úloha se zabývá výpočtem faktoriálu. Nejdříve bez pomoci knihoven a poté pomocí
knihovny \textit{SymPy}. Časová náročnost je následující:
\justify
Výpočet (Python):  1.8854195330059156\\
Výpočet (SymPy):  0.1536074520117836
\\
\subsubsection{2. Výpočet sin(x)}
Tentokrát porovnáváme 3 knihovny - \textit{Math}, \textit{NumPy} a \textit{SymPy}.
\justify
Čas výpočtu (Python):  0.00030469500052277\\
Čas výpočtu (NumPy):   0.0011716900044120848\\
Čas výpočtu (Sympy):   0.6211337630084017
\justify
Python s pomocí knihovny math je zde nejrychlejší. Po něm je NumPy a poté SymPy.\\
\subsubsection{3. Násobení matic}
Tato úloha ukazuje  efektivitu násobení matic o velikosti 3x3 pomocí knihovny \textit{NumPy}.
\justify
Čas výpočtu (Python):  0.00018488299974706024\\
Čas výpočtu (NumPy):  0.00012147700181230903
\justify
Výsledky ukazují, že je knihovna NumPy výrazně rychlejší. Python využívá tři vnořené smyčky k výpočtu, zatímco NumPy používá vektorové operace pro rychlejší výpočet.\\
\subsubsection{4. Výpočet faktoriálu}
Tato úloha se zabývá výpočtem faktoriálu. Nejdřív pomocí Pythonu a poté pomocí
knihovny \textit{Math}.
\justify
Čas výpočtu (Python):  0.15173245201003738\\
Čas výpočtu (Math):  0.01574059799895622\\
\subsection{5. Aritmetický průměr}
Tato úloha ukazuje délku výpočtu aritmetického průměru seznamu čísel o délce\\3 524 287 čísel. 
\justify
Čas výpočtu (Python):  0.031385743990540504\\
Čas výpočtu (NumPy):   0.009323335994849913
\justify
Ve výsledcích můžeme vidět, že výpočet za pomoci \textit{NumPy} je výrazně rychlejší. Python musí projít celý seznam a získat sumu prvků, zatímco \textit{NumPy} vypočítá průměr pomocí své vestavěné funkce mean().\\

	
\end{document}
